%% content.tex
%%

%% ==============
\chapter{Grundlagen}
\label{ch:Grundlagen}
%% ==============


%% ===========================
\section{OLED}
\label{ch:Grundlagen:sec:OLED}
%% ===========================

\chapter{Herstellung}
\label{ch:Herstellung}
Grundlage f�r die Herstellung von OLEDs bildet ein Glassubstrat, dass mit einer \todo{XX? nm} dicken Schicht ITO beschichtet ist. Aus diesem Substrat wurden mit dem Glasschneider zun�chst 8 16~x~16~mm$^2$ gro�e Proben geschnitten. Anschlie�end wurde an einer Seite ca. ein Viertel der ITO Beschichtung wegge�tzt. Hierzu wird 3/4 der Proben mit einem Klebestreifen abgeklebt. Anschlei�end werden sie 7~min in ein Bad aus 37~\%iger Salzs�ure gegeben, das zus�tzlich noch mit einer Spartelspitze Zinkpulver versetzt wird. Nach dem �tzprozess werden die Proben mit Wasser abgesp�hlt und der Klebestreifen abgezogen. Die Vorbereitung der Proben wurde im Vorfeld des Versuchs durch Tobias Bocksrocker durchgef�hrt.

Die Proben werden nun in ein Becherglas mit Aceton gegeben und 10~min in ein Ultraschallbad gestellt. Danach werden die Proben in Isopropanol gestellt und ebenfalls im Ultraschallbad weitere 10~min gereinigt. Durch die Reinigung mit Aceton werden Fette und Schmutz von der Oberfl�che der Proben gel�st. Im Isopropanol l�sen sich weitere R�ckst�nde, die sich in Aceton nicht l�sen. Au�erdem werden durch das Isopropanol auch Acetonreste gel�st. Nach der Reinigung durch Isopropanol werden die Proben mit Stickstoff trockengeblasen und anschlie�end f�r 2~min in einen Plasmaverascher gegeben. Hierin sind die Proben einem Sauerstoffplasma ausgesetzt. Die darin enthaltenen Sauerstoffradikale f�hren zum einen dazu, dass organische Reste auf der Oberfl�che verbrennen. Zum anderen lagern sich auch Sauerstoffatome an der Oberfl�che des ITO an. Diese sogenannte Aktivierung f�hrt dazu, dass die Oberfl�che polar wird, und sich somit Polare L�sungen auf der Oberfl�che besser verteilen. Au�erdem wird auch die Austrittsarbeit des ITO erh�ht. Nach dem Plasmaveraschen werden die Proben mit einem Wasserfesten Folienstift auf der Glasseite durchnummeriert, damit sie im Folgenden unterscheidbar sind.

Als n�chstes wurde eine PEDOT:PSS-Wasser-L�sung im Verh�lltnis 1:1 hergestellt.\todo{evtl PSS erw�hnen}  Hierzu wurden zun�chst 750~ml PEDOT:PSS mit einem Membranfilter gefiltert und anschlie�end 750~ml Reinstwasser hinzugegeben. Um die L�sung zu durchmischen und Polymerklumpen zu vermeiden , wird die L�sung in ein Ultraschallbad mit niedriger Leistung (110~W) gegeben. Die Polymerl�sung "`Superyellow"' wurde bereits einen Tag vorher von Tobias Bocksrocker hergestellt, da diese mindestens 24~h zum durchmischen ben�tigt.

Da das Polymer "`Superyellow"' Sauerstoffempfindlich ist, wird die weitere Verarbeitung in einer Glove-Box mit Stickstoffatmosph�re durchgef�hrt. Die Proben wurden zusammen mit der PEDOT:PSS-L�sung in die Glove-Box geschleust. Hierzu wird die Schleusenkammer drei mal evakuiert und mit Stickstoff geflutet um zu vermeiden, dass Sauerstoff mit in die Glove-Box geschleust wird. Nun wird per Spin-Coating eine Schicht aus PEDOT:PSS aufgebracht. Hierzu wurden mit einer \todo{Eppendorf-Pipette??} 150~$\upmu$l PEDOT:PSS-L�sung gleichm��ig auf dem Substrat verteilt.