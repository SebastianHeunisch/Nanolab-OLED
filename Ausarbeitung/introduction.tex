%% introduction.tex
%%

%% ==============================
\chapter{Einleitung}
\label{ch:Introduction}
%% ==============================
Organische Leuchtdioden (engl. \textit{organic light emitting diode} - OLED) werden Heute bereits in zahlreichen Ger�ten verwendet. Seit Anfang der 90er Jahre haben sie sich rasch entwickelt. Man kann mit ihnen inzwischen optoelektronische Anzeige mit gro�em Farbraum und hoher Aufl�sung herstellen. So findet man zahlreiche Smartphones auf dem Markt die OLEDs als Displaytechnologie verwenden.
Auch als Leuchtmittel sind OLEDs eine vielversprechende Technologie. Im Gegensatz zu anorganischen LEDs sind OLEDs keine Punktlichtquelle, sondern es lassen sich leuchtende Fl�chen herstellen. Dies er�ffnet vielf�ltige neue Gestaltungsm�glichkeiten in der Allgemeinbeleuchtung. Au�erdem besitzen OLEDs theoretisch eine Lichtausbeute, die mit anorganischen LEDs vergleichbar ist. So lassen sich in der Beleuchtungstechnik Energie und damit auch Kosten sparen.
Die Herausforderung bei der Herstellung von OLEDs besteht darin die Lebensdauer zu verl�ngern, die Herstellungskosten weiter zu senken und die Lichtausbeute zu steigern. Daher sind OLEDs auch am LTI Gegenstand intensiver Forschung. Darum wird auch im Labor Nanotechnologie ein Versuch durchgef�hrt, indem eine Einf�hrung in die Herstellung von OLEDs gegeben wird.
% 
% Nach intensiver Forschung sind die Preise f�r die Herstellung von OLEDs sind in den letzten Jahren stark gefallen. Hieran wird weiter intensiv geforscht.
% 
% Auch als Leuchtmittel ist sind OLEDs eine vielversprechende Technologie. So lassen sich mit ihnen effiziente Fl�chenstrahler herstellen.

